\documentclass[]{exam}
\usepackage{lmodern}
\usepackage{amssymb,amsmath}
\usepackage{ifxetex,ifluatex}
\usepackage{fixltx2e} % provides \textsubscript
\ifnum 0\ifxetex 1\fi\ifluatex 1\fi=0 % if pdftex
  \usepackage[T1]{fontenc}
  \usepackage[utf8]{inputenc}
\else % if luatex or xelatex
  \ifxetex
    \usepackage{mathspec}
    \makeatletter % undo the wrong changes made by mathspec
    \let\RequirePackage\original@RequirePackage
    \let\usepackage\RequirePackage
    \makeatother
  \else
    \usepackage{fontspec}
  \fi
  \defaultfontfeatures{Ligatures=TeX,Scale=MatchLowercase}
\fi
% use upquote if available, for straight quotes in verbatim environments
\IfFileExists{upquote.sty}{\usepackage{upquote}}{}
% use microtype if available
\IfFileExists{microtype.sty}{%
\usepackage{microtype}
\UseMicrotypeSet[protrusion]{basicmath} % disable protrusion for tt fonts
}{}
\usepackage{hyperref}
\hypersetup{unicode=true,
            pdftitle={Homework 3},
            pdfborder={0 0 0},
            breaklinks=true}
\urlstyle{same}  % don't use monospace font for urls
\IfFileExists{parskip.sty}{%
\usepackage{parskip}
}{% else
\setlength{\parindent}{0pt}
\setlength{\parskip}{6pt plus 2pt minus 1pt}
}
\setlength{\emergencystretch}{3em}  % prevent overfull lines
\providecommand{\tightlist}{%
  \setlength{\itemsep}{0pt}\setlength{\parskip}{0pt}}
\setcounter{secnumdepth}{0}
% Redefines (sub)paragraphs to behave more like sections
\ifx\paragraph\undefined\else
\let\oldparagraph\paragraph
\renewcommand{\paragraph}[1]{\oldparagraph{#1}\mbox{}}
\fi
\ifx\subparagraph\undefined\else
\let\oldsubparagraph\subparagraph
\renewcommand{\subparagraph}[1]{\oldsubparagraph{#1}\mbox{}}
\fi

\usepackage{booktabs}

\title{Homework 3}

\usepackage[margin=.75in]{geometry}
\RequirePackage{amsmath}
\usepackage{amssymb}
\usepackage{textcase}
\usepackage{soul}
\usepackage{indentfirst}

\newcommand{\eps}{\varepsilon}
\newcommand{\kron}{\otimes}
\DeclareMathOperator{\diag}{diag}
\DeclareMathOperator{\trace}{trace}
\DeclareMathOperator{\tvec}{vec}
\DeclareMathOperator{\rank}{rank}
\DeclareMathOperator{\tspan}{span}
\DeclareMathOperator*{\minimize}{minimize}
\DeclareMathOperator*{\maximize}{maximize}
\DeclareMathOperator{\subjectto}{subject\ to}

\newcommand{\mat}[1]{\boldsymbol{#1}}
\renewcommand{\vec}[1]{\boldsymbol{\mathrm{#1}}}
\newcommand{\vecalt}[1]{\boldsymbol{#1}}

\newcommand{\conj}[1]{\overline{#1}}

\newcommand{\normof}[1]{\|#1\|}
\newcommand{\onormof}[2]{\|#1\|_{#2}}

\newcommand{\MIN}[2]{\begin{array}{ll} \displaystyle \minimize_{#1} & {#2} \end{array}}
\newcommand{\MINone}[3]{\begin{array}{ll} \displaystyle \minimize_{#1} & {#2} \\ \subjectto & {#3} \end{array}}
\newcommand{\OPTone}{\MINone}
\newcommand{\MINthree}[5]{\begin{array}{ll} \displaystyle \minimize_{#1} & {#2} \\ \subjectto & {#3} \\ & {#4} \\ & {#5} \end{array}}

\newcommand{\MAX}[2]{\begin{array}{ll} \displaystyle \maximize_{#1} & {#2} \end{array}}
\newcommand{\MAXone}[3]{\begin{array}{ll} \displaystyle \maximize_{#1} & {#2} \\ \subjectto & {#3} \end{array}}


\newcommand{\itr}[2]{#1^{(#2)}}
\newcommand{\itn}[1]{^{(#1)}}

\newcommand{\prob}{\mathbb{P}}
\newcommand{\probof}[1]{\prob\left\{ #1 \right\}}

\newcommand{\pmat}[1]{\begin{pmatrix} #1 \end{pmatrix}}
\newcommand{\bmat}[1]{\begin{bmatrix} #1 \end{bmatrix}}
\newcommand{\spmat}[1]{\left(\begin{smallmatrix} #1 \end{smallmatrix}\right)}
\newcommand{\sbmat}[1]{\left[\begin{smallmatrix} #1 \end{smallmatrix}\right]}

\newcommand{\RR}{\mathbb{R}}
\newcommand{\CC}{\mathbb{C}}

\providecommand{\eye}{\mat{I}}
\providecommand{\mA}{\ensuremath{\mat{A}}}
\providecommand{\mB}{\ensuremath{\mat{B}}}
\providecommand{\mC}{\ensuremath{\mat{C}}}
\providecommand{\mD}{\ensuremath{\mat{D}}}
\providecommand{\mE}{\ensuremath{\mat{E}}}
\providecommand{\mF}{\ensuremath{\mat{F}}}
\providecommand{\mG}{\ensuremath{\mat{G}}}
\providecommand{\mH}{\ensuremath{\mat{H}}}
\providecommand{\mI}{\ensuremath{\mat{I}}}
\providecommand{\mJ}{\ensuremath{\mat{J}}}
\providecommand{\mK}{\ensuremath{\mat{K}}}
\providecommand{\mL}{\ensuremath{\mat{L}}}
\providecommand{\mM}{\ensuremath{\mat{M}}}
\providecommand{\mN}{\ensuremath{\mat{N}}}
\providecommand{\mO}{\ensuremath{\mat{O}}}
\providecommand{\mP}{\ensuremath{\mat{P}}}
\providecommand{\mQ}{\ensuremath{\mat{Q}}}
\providecommand{\mR}{\ensuremath{\mat{R}}}
\providecommand{\mS}{\ensuremath{\mat{S}}}
\providecommand{\mT}{\ensuremath{\mat{T}}}
\providecommand{\mU}{\ensuremath{\mat{U}}}
\providecommand{\mV}{\ensuremath{\mat{V}}}
\providecommand{\mW}{\ensuremath{\mat{W}}}
\providecommand{\mX}{\ensuremath{\mat{X}}}
\providecommand{\mY}{\ensuremath{\mat{Y}}}
\providecommand{\mZ}{\ensuremath{\mat{Z}}}
\providecommand{\mLambda}{\ensuremath{\mat{\Lambda}}}
\providecommand{\mSigma}{\ensuremath{\mat{\Sigma}}}
\providecommand{\mPbar}{\bar{\mP}}

\providecommand{\ones}{\vec{e}}
\providecommand{\va}{\ensuremath{\vec{a}}}
\providecommand{\vb}{\ensuremath{\vec{b}}}
\providecommand{\vc}{\ensuremath{\vec{c}}}
\providecommand{\vd}{\ensuremath{\vec{d}}}
\providecommand{\ve}{\ensuremath{\vec{e}}}
\providecommand{\vf}{\ensuremath{\vec{f}}}
\providecommand{\vg}{\ensuremath{\vec{g}}}
\providecommand{\vh}{\ensuremath{\vec{h}}}
\providecommand{\vi}{\ensuremath{\vec{i}}}
\providecommand{\vj}{\ensuremath{\vec{j}}}
\providecommand{\vk}{\ensuremath{\vec{k}}}
\providecommand{\vl}{\ensuremath{\vec{l}}}
\providecommand{\vm}{\ensuremath{\vec{m}}}
\providecommand{\vn}{\ensuremath{\vec{n}}}
\providecommand{\vo}{\ensuremath{\vec{o}}}
\providecommand{\vp}{\ensuremath{\vec{p}}}
\providecommand{\vq}{\ensuremath{\vec{q}}}
\providecommand{\vr}{\ensuremath{\vec{r}}}
\providecommand{\vs}{\ensuremath{\vec{s}}}
\providecommand{\vt}{\ensuremath{\vec{t}}}
\providecommand{\vu}{\ensuremath{\vec{u}}}
\providecommand{\vv}{\ensuremath{\vec{v}}}
\providecommand{\vw}{\ensuremath{\vec{w}}}
\providecommand{\vx}{\ensuremath{\vec{x}}}
\providecommand{\vy}{\ensuremath{\vec{y}}}
\providecommand{\vz}{\ensuremath{\vec{z}}}
\providecommand{\vpi}{\ensuremath{\vecalt{\pi}}}

\providecommand{\vlambda}{\ensuremath{\vecalt{\lambda}}}


\sodef\allcapsspacing{\upshape}{0.15em}{0.65em}{0.6em}%

\makeatletter
\def\maketitle{%
\par
\hrule height 0.75pt\vspace{1ex}
\par\noindent
\begin{minipage}{0.5\textwidth}
\scshape
purdue university $\cdot$ cs 51500 \\
matrix computations
\end{minipage}
\begin{minipage}{0.5\textwidth}
\raggedleft
\MakeTextUppercase{\allcapsspacing{\@title}}\\[0.2ex]
\textit{\@author}\\[0.2ex]
\textit{\@date}
\end{minipage}
\par\vspace{1ex}
\hrule height 1pt
\vspace{2ex}
\par
}
\makeatother

\author{Jinen Setpal}
\title{Lecture Notes}
% auto generate a title
% \AtBeginDocument{\maketitle}

\title{Homework 3 Submission}

% Question header formatting
\qformat{\hfill \textbf{Problem \thequestion} \hfill}

\DeclareMathOperator{\Tr}{Tr}
\DeclareMathOperator{\Cov}{Cov}
\DeclareMathOperator{\Concat}{Concat}
\DeclareMathOperator*{\argsup}{arg\,sup}
\DeclareMathOperator*{\arginf}{arg\,inf}
\DeclareMathOperator*{\argmax}{arg\,max}
\DeclareMathOperator*{\argmin}{arg\,min}
\DeclareMathOperator*{\indep}{\perp \!\!\! \perp}
\setcounter{MaxMatrixCols}{20}

\newcommand{\divider}{\line(1,0){\textwidth}}

\allowdisplaybreaks

% Package for enumerated lists using letters
\usepackage{enumitem}

% Algorithms
\usepackage{algpseudocode}
\usepackage{algorithm}
\usepackage{nicefrac}
\usepackage{graphicx}
\usepackage{fancyvrb,fvextra}
\usepackage[usenames,dvipsnames]{xcolor}
\usepackage{listings}
\usepackage{hyperref}

%%
%% Julia definition (c) 2014 Jubobs
%%
\lstdefinelanguage{Julia}%
  {morekeywords={abstract,break,case,catch,const,continue,do,else,elseif,%
      end,export,false,for,function,immutable,import,importall,if,in,%
      macro,module,otherwise,quote,return,switch,true,try,type,typealias,%
      using,while},%
   sensitive=true,%
   alsoother={\$},%
   morecomment=[l]\#,%
   morecomment=[n]{\#=}{=\#},%
   morestring=[s]{"}{"},%
   morestring=[m]{'}{'},%
}[keywords,comments,strings]%

\lstset{%
    language         = Julia,
    basicstyle       = \ttfamily,
    keywordstyle     = \bfseries\color{blue},
    stringstyle      = \color{magenta},
    commentstyle     = \color{ForestGreen},
	breaklines=true,
    showstringspaces = false,
}

\begin{document}
\maketitle

\hypertarget{problem-0-homework-checklist}{%
\subsection{Checklist}\label{problem-0-homework-checklist}}

\begin{enumerate}
	\item Cross-checked independent work with Kunal Kapur.
	\item No use of AI tools.
	\item Code is included!
\end{enumerate}

\begin{questions}

\question
\hfill

We have:
\begin{gather*}
	\mA = \bmat{\mA_1 & \mA_2 \\ \mA_3 & \mA_4}, \quad \vx = \bmat{\vx_1 \\ \vx_2}, \quad \vb = \bmat{\vb_1 \\ \vb_2}
\end{gather*}
Which decomposes into the following two systems:
\begin{gather*}
	\mA_1\vx_1 + \mA_2 \vx_2 = \vb_1 \\
	\mA_3\vx_1 + \mA_4 \vx_2 = \vb_2
\end{gather*}
We can assume $\vx_2$ given and solve for $\vx_1$:
\begin{align*}
	&\vx_1 = \mA_1^{-1} (\vb_1 - \mA_2 \vx_2) \\
	&\implies \mA_3\mA_1^{-1} (\vb_1 - \mA_2 \vx_2) + \mA_4 \vx_2 = \vb_2 \\
	&\implies \mA_3\mA_1^{-1} \vb_1 - \mA_3\mA_1^{-1} \mA_2 \vx_2 + \mA_4 \vx_2 = \vb_2 \\
	&\implies (\mA_4 - \mA_3\mA_1^{-1} \mA_2) \vx_2 = \vb_2 - \mA_3\mA_1^{-1} \vb_1 
\end{align*}

Now, we have inverses, but we can just recurse on our solver until we have a trivial system.
	\paragraph{Code}
	\begin{lstlisting}
using SparseArrays, LinearAlgebra, Plots

function solve(A::Matrix, B::Matrix)
	n, k = size(B)
	factor = div(n, 2)

	if n == 1
		return B ./ A
	else
		A_1 = A[1:factor, 1:factor]
		A_2 = A[1:factor, end-factor+1:end]
		A_3 = A[end-factor+1:end, 1:factor]
		A_4 = A[end-factor+1:end, end-factor+1:end]
		
		x_2 = solve(A_4 - A_3 * solve(A_1, A_2), B[end-factor+1:end, :] - A_3 * solve(A_1, B[1:factor, :]))
		x_1 = solve(A_1, B[1:factor, :] - A_2 * x_2)
		return vcat(x_1, x_2)
	end
end

Z = randn(2^4, 2^4)
A = Z'Z
b = ones(size(Z)[1], 1)

x_hat = solve(A, b)
println(norm(x_hat - A\b))
	\end{lstlisting}
	\paragraph{Output}
	\begin{lstlisting}
7.166880891681717e-9
	\end{lstlisting}

\newpage
\question
\hfill

\begin{enumerate}[label=\arabic*.]
	\item Each step of Cholesky Factorization does two things: a) it preserves the symmetry and positive definiteness of a matrix, and b) it reduces the size of the matrix from $n \times n$ to $n - 1 \times n - 1$. This reveals a simple application of induction, given it holds trivially for a $1 \times 1$ case (just the square root of arbitrary input given positive definiteness), we can assume it works for the $k \times k$ case as our inductive hypothesis, and the inductive step reduces a $(k + 1) \times (k + 1)$ case to the $k \times k$ case. We therefore have a Cholesky factorization for any symmetric positive definite matrix of arbitrary dimension.
	\item The positive definiteness condition states that, for a matrix $\mA$, it holds that:
		$$ \vx^T \mA \vx > 0 \quad \forall \vx \neq 0 \qquad \mathrm{and} \qquad \vx^T \mA^T \vx > 0 \quad \forall \vx \neq 0$$
		The second equivalent statement is just the transpose, which holds because the left term is a scalar. We can simplify the statement by adding the two terms to get an equivalent condition:
		\begin{align*}
			&\vx^T \mA \vx + \vx^T \mA^T \vx > 0 & \forall \vx \neq 0 \\
			\iff &(\vx^T \mA + \vx^T \mA^T) \vx > 0 & \forall \vx \neq 0 \\
			\iff &(\vx^T (\mA + \mA^T)) \vx > 0 & \forall \vx \neq 0 \\
			\iff &\vx^T (\mA + \mA^T) \vx > 0 & \forall \vx \neq 0
		\end{align*}
		This is awesome, because $\mB := \mA + \mA^T$ is a symmetric matrix. Proof:
		\begin{align*}
			\mB_{ij} = \sum^n_{k=1} \mA_{i, k} + \mA^T_{k,j} = \sum^n_{k=1} \mA^T_{k, i} + \mA_{j,k} = \mB_{ji} \qquad \forall i,j
		\end{align*}

		If $\alpha < 0$ in any step of Cholesky factorization of $\mB$, then our matrix $\mB$ is not positive definite, which is an equivalent condition for $\mA$ not being positive definite. This allows us to modify the code from the textbook:
		\paragraph{Code}
\begin{lstlisting}[language=julia]
using SparseArrays, LinearAlgebra

# important: copied and modified from textbook
function check_definiteness(X::Matrix, check_positive::Bool)  # if check_positive = true, check if positive definite, else check if negative definite
	A = copy(Float64.(X) + X')
	if !check_positive
		A = -A
	end

	n = size(A,1)
	F = Matrix(1.0I,n,n)
	d = zeros(n)
	for i=1:n-1
		alpha = A[i,i]
		if alpha < 0
			return false
		end
		d[i] = sqrt(alpha)
		F[i+1:end,i] = A[i+1:end,i]/alpha
		A[i+1:end, i] -= A[i+1:end, i] * A[i, i+1:end]'/alpha
	end
	return true
end

A = randn(10, 10)
PD = A'*A
println("Sanity Check 1 (expect true): ", check_definiteness(PD, true))
println("Sanity Check 2 (expect false): ", check_definiteness(A, true))


function map_index(i::Integer, j::Integer, n::Integer)
	if 1 < i < n+1 && 1 < j < n+1
		return 4n + (i - 2)*(n-1) + j-1
	elseif i == 1
		return j
	elseif i == n+1
		return n + 1 + j
	elseif j == 1
		return 2(n+1) + i - 1
	elseif j == n+1
		return 2(n+1) + n - 2 + i
	end
end

function laplacian(n::Integer, f::Function)
	A = sparse(1I, (n+1)^2, (n+1)^2)
	A[diagind(A)[4n+1:end]] .= -4

	fvec = zeros((n+1)^2)

	global row_index = 4n + 1
	for i in 2:n
		for j in 2:n
			A[row_index, map_index(i-1, j, n)] = 1
			A[row_index, map_index(i+1, j, n)] = 1
			A[row_index, map_index(i, j-1, n)] = 1
			A[row_index, map_index(i, j+1, n)] = 1
			fvec[row_index] = f(i, j)

			global row_index += 1
		end
	end

	return A, fvec/n^2
end

n = 10
A, fv = laplacian(n, (x, y) -> 1)
println()
println("Laplacian with boundary conditions (Positive): ", check_definiteness(Matrix(A), true))
println("Laplacian with boundary conditions (Negative): ", check_definiteness(Matrix(A), false))

A_smol = A[4n+1:end,4n+1:end]
println()
println("Laplacian without boundary conditions (Positive): ", check_definiteness(Matrix(A_smol), true))
println("Laplacian without boundary conditions (Negative): ", check_definiteness(Matrix(A_smol), false))
\end{lstlisting}
		\paragraph{Code}
\begin{lstlisting}
Sanity Check 1 (expect true): true
Sanity Check 2 (expect false): false

Laplacian with boundary conditions (Positive): false
Laplacian with boundary conditions (Negative): false

Laplacian without boundary conditions (Positive): false
Laplacian without boundary conditions (Negative): true
\end{lstlisting}
\end{enumerate}
Thus we conclude that Poisson's equation from homework 1 \textit{without boundary conditions} is \textbf{negative definite}.

\newpage
\question
\hfill

We have:
\begin{gather*}
	\mA = \bmat{\alpha_1 & \beta_1 \\ \beta_1 & \alpha_2 & \beta_2 \\ & \ddots & \ddots & \ddots \\ & & \beta_{n-2} & \alpha_{n-1} & \beta_{n-1} \\ & & & \beta_{n-1} & \alpha_n} \\
\end{gather*}

\begin{enumerate}[label=\arabic*.]
	\item Since $\mA_{n-1}$ is tridiagonal, we have that:
		\begin{gather*}
			(\mA_{n-1})_{ij} = (\mL_{n-1} \mL_{n-1}^T)_{ij} = 0 \qquad \forall i,j : j - 1 \leq i \leq j + 1 \\
			(\mL_{n-1} \mL_{n-1}^T)_{ij} = \sum^n_{k=1} (\mL_{n-1})_{ik} (\mL_{n-1}^T)_{kj}
			= \sum^n_{k=1} (\mL_{n-1})_{ik} (\mL_{n-1})_{jk} = \langle \mL_{i, :}, \mL_{j, :} \rangle
		\end{gather*}
		Let $[n] = \{1, 2, \ldots, n\}$.
		Since $\mA_{n-1}$, is positive definite, we have that $(\mA_{n-1})_{ii} > 0~\forall i \in [n-1]$. Therefore,
		\begin{gather*}
			(\mA_{n-1})_{ii} = \sum^{n-1}_{k=1} (\mL_{n-1})_{ik}^2 = \| L_{i, :} \|^2_2 > 0 \qquad \forall i \in [n - 1]
		\end{gather*}
		For $i=1$, we have:
		\begin{gather*}
			(\mA_{n-1})_{11} = \sum^{n-1}_{k=1} (\mL_{n-1})_{1k} (\mL_{n-1})_{k1} = (\mL_{n-1})_{11} (\mL_{n-1})_{11} = (\mL_{n-1})_{11}^2 > 0
		\end{gather*}
		Thus $(\mL_{n-1})_{11} > 0$. We can use this to show that $(\mL_{n-1})_{i, j} = 0~\forall i > j + 1$ for $j = 1$:
		\begin{gather*}
			(\mA_{n-1})_{i1} = 0 = \langle (\mL_{n-1})_{i, :}, (\mL_{n-1})_{1, :} \rangle = (\mL_{n-1})_{i1} (\mL_{n-1})_{11}
		\end{gather*}
		Since $(\mL_{n-1})_{11} > 0$, it must hold that $(\mL_{n-1})_{i1} = 0$. Letting $(\mL_{n-1})_{21}$ remain arbitrary, we've comprehensively characterized the first column. We can extend this to arbitrary columns next.

		\textbf{Inductive Hypothesis:} Given column $k$, $(\mL_{n-1})_{j,j} > 0$ and $(\mL_{n-1})_{i,j} = 0~\forall i > j + 1~\forall j \leq k$.

		\textbf{Inductive Step:} T.P.T.~For column $k+1$, $(\mL_{n-1})_{k+1,k+1} > 0$ and $(\mL_{n-1})_{i,k+1} = 0~\forall i > k + 2$.

		We can use the above result to show that $(\mL_{n-1})_{k + 1, k+1} > 0$:
		\begin{align*}
			(\mA_{n-1})_{k+1, k+1} &= \langle (\mL_{n-1})_{k + 1, :}, (\mL_{n-1})_{k + 1, :} \rangle  > 0\\
			&= \sum^{(k+1)-1}_{l=1} (\mL_{n-1})_{k + 1, l}^2 > 0 \\
			&= (\mL_{n-1})_{k + 1, k+1}^2 + \sum^{k-1}_{l=1} (\mL_{n-1})_{k + 1, l}^2 > 0 \\
			&= (\mL_{n-1})_{k + 1, k+1}^2 + 0 > 0 \\
			(\mA_{n-1})_{k+1, k+1} &= (\mL_{n-1})_{k + 1, k+1}^2 > 0
		\end{align*}

		Finally, we can show that $(\mL_{n-1})_{i,k + 1} = 0~\forall i > k + 2$:
		\begin{align*}
			(\mA_{n-1})_{i, k+1} &= 0 = \langle (\mL_{n-1})_{i, :}, (\mL_{n-1})_{k+1, :} \rangle = 0 + 0 + \ldots + 0 + (\mL_{n-1})_{i, k+1} (\mL_{n-1})_{k+1, k+1}
		\end{align*}
		Since $(\mL_{n-1})_{k + 1, k+1} > 0$, it holds that $(\mL_{n-1})_{i, k+1} = 0$.

		Therefore, we have proven that $\mL_{n-1}$ has a non-zero diagonal, and that it is sparse; it is populated only by the diagonal and subdiagonal elements.
	\item We can setup this question by unpacking a single step of Cholesky Decomposition:
		\begin{gather*}
			\mA = \bmat{\alpha_1 & \beta_1 \ve_1^T \\ \beta_1 \ve_1 & \mA_{n - 1}} = \mC \mC^T = \bmat{\gamma & 0 \\ \vc & \mC_{1}} \bmat{\gamma & \vc^T \\ 0 & \mC_{1}^T} = \bmat{\gamma^2 & \gamma \vc^T \\ \gamma \vc & \vc \vc^T + \mC_1 \mC_1^T}
		\end{gather*}
		So, we have $\gamma = \sqrt{\alpha},~\vc = (\nicefrac{\beta_1}{\gamma}) \ve_1$. Substituting the given equalities, we have:
		\begin{gather*}
			\mA_{n-1} = \mL_{n-1} \mL_{n-1}^T = \vc \vc^T + \mC_1 \mC_1^T = \left(\frac{\beta_1}{\sqrt{\alpha}}\right)^2 \ve_{11} + \mC_1 \mC_1^T \\
			\implies \mC_1 \mC_1^T = \mL_{n-1} \mL_{n-1}^T - \frac{\beta_1^2}{\alpha_1} \ve_{11}
		\end{gather*}
		We need the Cholesky Factorization for the right hand term. We can integrate element subtraction maintaining the sparse structure of $\mL_{n-1}$:
		\begin{gather*}
			(\mC_1)_{11}^2 = \langle (\mC_1)_{1, :}, (\mC_1)_{1, :} \rangle = (\mC_1 \mC_1^T)_{11} = \langle (\mL_{n-1})_{1,:}, (\mL_{n-1})_{1,:} \rangle - \frac{\beta_1^2}{\alpha_1} = (\mL_{n-1})_{11}^2 - \frac{\beta_1^2}{\alpha_1} \\
			\therefore (\mC_1)_{11} = \sqrt{(\mL_{n-1})_{11}^2 - \frac{\beta_1^2}{\alpha_1}}
		\end{gather*}
		Here, we only pick the positive solution since we have shown in the previous question that the diagonal elements are each $> 0$. The other element affected by this change is $(\mC_1 \mC_1^T)_{21}$:
		\begin{gather*}
			(\mC_1)_{11} (\mC_1)_{21} = \langle (\mC_1)_{1, :}, (\mC_1)_{2, :} \rangle = (\mC_1 \mC_1^T)_{21} = \langle (\mL_{n-1})_{1, :}, (\mL_{n-1})_{2, :} \rangle = (\mL_{n-1})_{11} (\mL_{n-1})_{21} \\
			\therefore (\mC_1)_{21} = \frac{(\mL_{n-1})_{11} (\mL_{n-1})_{21}}{(\mC_1)_{11}} = \frac{(\mL_{n-1})_{11} (\mL_{n-1})_{21}}{\sqrt{(\mL_{n-1})_{11}^2 - \frac{\beta_1^2}{\alpha_1}}}
		\end{gather*}
		The rest of the matrix is unaffected by the single-element change (because we have that $\mC$ is only populated by the diagonal and sub-diagonal, $(\mC_1)_{j1} = 0~\forall j > 2$ and so the dot product breakdown used earlier can ignore $(\mC_1)_{11}$) and is identical to $\mL_{n-1}$. Thus, we have obtained $\gamma, \vc, \mC_1$ which together constitutes the Cholesky factorization of $\mA$ following the first equation.
	\item We can apply the algorithm mathematically described above using a constant set of operations:
		\paragraph{Code}
		\begin{lstlisting}
function tridiag_cholesky(A::Matrix)
	L = spzeros(size(A))
	n, m = size(A)

	if n == 1
		return sqrt(A)
	else
		L[1, 1] = sqrt(A[1, 1])  # set gamma
		L[2, 1] = A[2, 1] / L[1, 1]  # rest of column vector is already zero

		# recurse till n = 1, and make relevant corrections
		L_1 = tridiag_cholesky(A[2:end, 2:end])
		L[2:end, 2:end] = L_1
		L[2, 2] = sqrt(L_1[1, 1]^2 - L[2, 1]^2)
		if n > 2
			L[3, 2] = L_1[1, 1] * L_1[2, 1] / L[2, 2]
		end
	end

	return L
end

A = Tridiagonal(ones(9), Vector(1:10), ones(9))
F, d = oracle(Matrix(A))
println("Decomposition Error: ", norm(A - F*F'))
		\end{lstlisting}
		\paragraph{Output}
		\begin{lstlisting}
Decomposition Error: 2.886579864025407e-15
		\end{lstlisting}
		This question is very cool!!
\end{enumerate}

\newpage
\question
\hfill

We have an undetermined linear system $(\mA - \lambda \mI)\vx = 0$. Running our elimination solver on this system results in the final 2x2 system of equations being linearly dependent. This propogates into a divide-by-zero error. For a full-rank system, our solver returns the {\it correct but undesired} trivial solution $\vx = 0$.

\paragraph{Output}
\begin{lstlisting}
3-element Vector{Float64}:
 NaN
 NaN
 NaN
\end{lstlisting}

By fixing a single variable from the solution $\vx_n := 1$, we can simply solve the linear system as normal. $\vx_{:n-1}$ `reacts' to $\vx_n$ and the overall result is an eigenvector.

\paragraph{Code}
\begin{lstlisting}
using LinearAlgebra

function solve1_pivot2_fixed(A::Matrix, b::Vector)
	m,n = size(A)
	@assert(m==n, "the system is not square")
	@assert(n==length(b), "vector b has the wrong length")
	if n==1
		@show(b)
		display(A)
		return [1.]   #  <-- this is the only functional change!
	else
		# let's make sure we have an equation
		# that we can eliminate!
		# let's try that again, where we pick the
		# largest magnitude entry!
		maxval = abs(A[1,1])
		newrow = 1
		for j=2:n
			if abs(A[j,1]) > maxval
				newrow = j
				maxval = abs(A[j,1])
			end
		end
		if maxval < eps(1.0)
			error("the system is singular")
		end
		@show newrow
		# swap rows 1, and newrow
		if newrow != 1
			tmp = A[1,:]
			A[1,:] .= A[newrow,:]
			A[newrow,:] .= tmp
			b[1], b[newrow] = b[newrow], b[1]
		end
		D = A[2:end,2:end]
		display(D)
		c = A[1,2:end]
		d = A[2:end,1]
		a = A[1,1]
		y = solve1_pivot2_fixed(D-d*c'/a, b[2:end]-b[1]/a*d)
		z = (b[1] - c'*y)/a
		return pushfirst!(y,z)
	end
end

A = [1 2 2; 0 2 1; -1 2 2]
lambda = 1

Y = A - lambda * I
b = zeros(size(Y)[1])

println("Valid problem? ", Bool(rank(A) - rank(Y)))
x_hat = solve1_pivot2_fixed(Y, b)[1:end]
println("Correct Solution? ", all((A * x_hat) ./ x_hat .== lambda))
\end{lstlisting}

\paragraph{Output}
\begin{lstlisting}
Valid problem? true
newrow = 3
2x2 Matrix{Int64}:
 1  1
 2  2
newrow = 2
1x1 Matrix{Float64}:
 1.0
b = [0.0]
1x1 Matrix{Float64}:
 0.0
Correct Solution? true
\end{lstlisting}

\newpage
\question
\hfill

We have two approaches to producing orthogonalizing $\va$:
\begin{gather*}
	\mH \va = \pm \| \va \| \ve_1 = \mG_{n-1} \mG_{n-2} \cdots \mG_1 \va
\end{gather*}

\textbf{Note:}
I got confused about how to interpret a `length $n$ vector', because rotations preserve vector length by default, and therefore both parts of this question have an identical answer. From reviewing the lecture, I instead assumed length as number of elements in the vector; this formulation is similar to the class discussion.

\begin{enumerate}[label=\arabic*.]
	\item For a 2-dimensional vector:
		\begin{gather*}
			\mG \va = \bmat{c & s \\ -s & c} \bmat{a_1 \\ a_2} = \bmat{c a_1 + s a_2 \\ -s a_1 + c a_2} = \bmat{\|\va\| \\ 0} \\
			\therefore c = \frac{a_1}{\| \va \|}, \quad s = \frac{a_2}{\| \va \|}
		\end{gather*}
		The orthogonal matrix from the Given's rotation is $\mG^T$:
		\begin{gather*}
			\va = \underbrace{\mG^T}_{\mQ} \underbrace{\| \va \| \ve_1}_{\mR}
			= \bmat{\frac{a_1}{\| \va \|} & -\frac{a_2}{\| \va \|} \\ \frac{a_2}{\| \va \|} & \frac{a_1}{\| \va \|}} \bmat{\|\va\| \\ 0}
			= \frac{1}{\| \va \|} \bmat{a_1 & -a_2 \\ a_2 & a_1} \bmat{\|\va\| \\ 0}
		\end{gather*}
		We can do something similar with Householder transformations:
		$$
			\mH = \mI - \frac{2 \vu \vu^T}{\|\vu\|^2_2} \quad \mathrm{where} \quad \vu = \va - \| \va \| \ve_1 = \bmat{a_1 - \|\va\| \\ a_2}
		$$
		\begin{align*}
			\therefore \mH
			&= \bmat{1 & 0 \\ 0 & 1} - \frac{2}{\vu^T \vu} \bmat{a_1 - \| \va \| \\ a_2} \bmat{a_1 - \| \va \| & a_2} \\
			&= \bmat{1 & 0 \\ 0 & 1} - \frac{2}{(a_1 - \sqrt{a_1^2 + a_2^2})^2 + a_2^2} \bmat{(a_1 - \| \va \|)^2 & (a_1 - \| \va \|) a_2  \\ (a_1 - \| \va \|) a_2 & a_2^2} \\
			&= \bmat{1 & 0 \\ 0 & 1} - \frac{2}{2(a_1^2 + + a_2^2 - \sqrt{a_1^2 + a_2^2})} \bmat{(a_1 - \| \va \|)^2 & (a_1 - \| \va \|) a_2  \\ (a_1 - \| \va \|) a_2 & a_2^2} \\
			&= \bmat{1 & 0 \\ 0 & 1} - \frac{1}{\|\va\|^2 - \|\va\|} \bmat{(a_1 - \| \va \|)^2 & (a_1 - \| \va \|) a_2  \\ (a_1 - \| \va \|) a_2 & a_2^2} \\
			&= \bmat{1 & 0 \\ 0 & 1} - \frac{1}{\|\va\|(1 - \|\va\|)} \bmat{(a_1 - \| \va \|)^2 & (a_1 - \| \va \|) a_2  \\ (a_1 - \| \va \|) a_2 & a_2^2}
		\end{align*}
		The orthogonal matrix from the Householder's rotation is $\mH^T$:
		\begin{gather*}
			\va = \underbrace{\mH^T}_{\mQ} \underbrace{\| \va \| \ve_1}_{\mR}
			= \left( \bmat{1 & 0 \\ 0 & 1} - \frac{1}{\|\va\|(1 - \|\va\|)} \bmat{(a_1 - \| \va \|)^2 & (a_1 - \| \va \|) a_2  \\ (a_1 - \| \va \|) a_2 & a_2^2} \right) \| \va \| \ve_1
		\end{gather*}
		$\mQ$ in case of Householder is symmetric, whereas Givens' produces a skew-symmetric matrix.
	\item For length 3 vectors, we have:
		\begin{gather*}
			\mG = \mG_2 \mG_1 = 
			\begin{pmatrix}c_2&s_2&0\\ -s_2&c_2&0\\ 0&0&1\end{pmatrix}\begin{pmatrix}c_1&0&s_1\\ 0&1&0\\ -s_1&0&c_1\end{pmatrix} = 
			\begin{pmatrix}c_2c_1&s_2&c_2s_1\\ -c_1s_2&c&-s_2s_1\\ -s_1&0&c_1\end{pmatrix} \\
				\va = \mG^T \| \va \| \ve_1 = \bmat{c_2 c_1 & -c_1 s_2 & -s_1 \\ s_2 & c & 0 \\ c_2 s_1 & -s_2 s_1 & c_1} \| \va \| \ve_1 
		\end{gather*}
		The orthogonal matrix in  this case is slightly sparse, but we can't really take advantage of this because $\mR$ is itself sparse and has a zero in the third index. For the case of Householder, we have:
		\begin{gather*}
			\mH = \mI - \frac{2 \vu \vu^T}{\|\vu\|^2_2} \quad \mathrm{where} \quad \vu = \va - \| \va \| \ve_1 = \bmat{a_1 - \|\va\| \\ a_2 \\ a_3} \\
			\begin{aligned}
				\therefore \mH
				&= \bmat{1 & 0 & 0 \\ 0 & 1 & 0 \\ 0 & 0 & 1} - \frac{2}{\vu^T \vu} \bmat{a_1 - \| \va \| \\ a_2 \\ a_3} \bmat{a_1 - \| \va \| & a_2 & a_3} \\
				&= \bmat{1 & 0 & 0 \\ 0 & 1 & 0 \\ 0 & 0 & 1} - \frac{2}{\vu^T \vu} \bmat{(a_1 - \| \va \|)^2 & (a_1 - \| \va \|) a_2 & (a_1 - \| \va \|) a_3 \\ (a_1 - \| \va \|) a_2 & a_2^2 & a_2 a_3 \\ (a_1 - \| \va \|) a_3 & a_3 a_2 & a_3^2} \\
				&= \bmat{1 & 0 & 0 \\ 0 & 1 & 0 \\ 0 & 0 & 1} - \frac{1}{\|\va\|(1 - \|\va\|)} \bmat{(a_1 - \| \va \|)^2 & (a_1 - \| \va \|) a_2 & (a_1 - \| \va \|) a_3 \\ (a_1 - \| \va \|) a_2 & a_2^2 & a_2 a_3 \\ (a_1 - \| \va \|) a_3 & a_3 a_2 & a_3^2}
			\end{aligned}
		\end{gather*}
		This allows us to make two observations: appending additional dimensions to Householder can be cheap, because we need to only add $2n - 1$ elements to a Householder matrix of one fewer dimensions. Further, since norms increase monotonically as the the overall magnitude of the orthogonal matrix is also more likelk y to reduce. If it doesn't, then it is likely that $\va$ is inherently sparse to begin with, and Given's rotations may produce an orthogonal matrix more cheaply.
\end{enumerate}

\end{questions}

\end{document}
